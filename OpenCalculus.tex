% Welcome to the TeX source of Open Calculus.
% This source (and all material from opencalculus.org
% is licensed under a Creative Commons Attribution-NonCommercial-ShareAlike 3.0 Unported License.
% (http://creativecommons.org/licenses/by-nc-sa/3.0/deed.en_US)
%
% As always, please email me directly at dixon@opencalculus.org with any comments, questions, suggestions, or concerns!

\documentclass[oneside]{article}

\usepackage{amsmath}
\usepackage{anysize}
\usepackage{hyperref}
\usepackage{graphicx}
\usepackage{bigstrut}

\marginsize{0.5in}{0.5in}{0.5in}{0.5in}

\linespread{1.1}

\begin{document}


\begin{titlepage}

\newcommand{\HRule}{\rule{\linewidth}{0.5mm}} 

\center

\textsc{\large \href{http://opencalculus.org}{opencalculus.org}}\\[1.5cm] 
\textsc{\Large \textit{A free, open-source text}}\\[0.5cm] 

\HRule \\[0.4cm]
{ \huge \bfseries Open Calculus}\\[0.4cm]
\HRule \\[1.5cm]

\Large \emph{Author:}\\
Dixon \textsc{Crews}\\[0.5cm]
\small \texttt{\href{mailto:dixon@opencalculus.org}{dixon@opencalculus.org}} \\ [3.5cm]

{\large Version 1.01}\\[1cm]
{\large March 2014}\\[3cm]

\vfill 

\end{titlepage}

\tableofcontents

\newpage

\section{Introduction}
\subsection{About this text}
\textbf{Welcome!} Before we start getting into the calculus, I thought it would be best to introduce both myself and this text. My name is Dixon Crews and I'm a student at North Carolina State University in Raleigh, N.C. majoring in Computer Science. I'm also minoring in Music Performance with a concentration in piano. I was first introduced to calculus in my senior year of high school in an AP Calculus BC course. I have found the study and applications of calculus to be utterly fascinating, and I have also seen first hand the need for a free and open-source calculus text to aid those currently enrolled in a college-level math course.

This text is not meant to replace a conventional calculus textbook. It is merely meant to serve as a second resource for students who may be struggling to grasp a difficult concept or want to see something expressed a different way. It should be apparent that I am not an authority on all things mathematical, and therefore this book should not be seen as the be all end all of calculus texts. Instead, we will approach the topics of college-level calculus courses with a reasonable but not exhaustive rigor that anyone may understand. At this time, I am only planning to cover what most colleges call Calculus I and II, mainly consisting of single variable differentiation and integration.

This is an \textit{ongoing, ever-changing} text. I welcome and encourage all suggestions, comments, and questions in the form of an email to \texttt{\href{mailto:dixon@opencalculus.org}{dixon@opencalculus.org}}. 

This text is licensed under a \href{http://creativecommons.org/licenses/by-nc-sa/3.0/deed.en_US}{Creative Commons Attribution-NonCommercial-ShareAlike 3.0 Unported License}.

\subsection{What is Calculus?}
\textbf{Calculus} is a branch of mathematics focused on limits, functions, derivatives, integrals, and infinite series. It has two major branches, differential calculus and integral calculus, which are related to the fundamental theorem of calculus. Calculus has widespread applications in science, economics, and engineering and can solve many problems for which algebra alone is insufficient. \footnote{\url{http://en.wikipedia.org/wiki/Calculus}}

\section{Limits}

\section{Differentiation}

\section{Integration}

\section{Sequences and Series}


\end{document}